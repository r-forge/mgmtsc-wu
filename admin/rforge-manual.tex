\documentclass[a4paper]{article}

% PREAMBLE
%\usepackage[austrian]{babel}
%\usepackage[utf8]{inputenc}
\usepackage[authoryear,round]{natbib}
\usepackage{graphicx}
\usepackage{hyperref}

\title{Rforge - User's manual v0.1}
\author{Stefan Theussl}
%\institute{Department of Statistics and Mathematics\newline
%Vienna University of Economics and Business Administration}
\date{\today}

\begin{document}

\maketitle

\begin{abstract}
\textbf{Rforge} is a service for the R community based on
\textbf{Gforge} which is a framework for collaborative software
development. There are various tools which allow the users to
communicate and organize their work. This manual is a quick guide to
this system.
\end{abstract}

\tableofcontents

\newpage
\section{Introduction}
\label{sec:intro}
As it is one of the keywords 2007, collaborative development becomes
more and more important. \textbf{Rforge} provides a framework based on
\textbf{Gforge} to the R Community which integrates various tools like
``subversion'' or mailinglists, etc. into one platform. It is a
place where all R developers and users can come together and exchange
their knowledge. For a detailed documentation
see~\cite{manual:gforge}. In this manual only the most important steps to get
started with \textbf{Rforge} are going to be described. 

\section{Registration}
\label{sec:registration}

\subsection{Registering a  new user}
To register a new user, click on the ``New Account'' link on the top
right side of the browser window.
Fill out the form (there are descriptions and hints for each field on
this site) and click on ``Submit'' afterwards. You get an email with
a URL to your specified email address. After clicking on this link
your account has been verified. Now you're able to login to the
website.
\newline
\textbf{Important note:} The tab ``My Page'' is the most important
page on \textbf{Rforge}. There you can configure your account, you can see your
project memberships and see the items, which have been assigned to you
(ie. bugs, feature requests and so forth).
\newline
Now you can start your own project (see section \ref{sec:newproject}
for details) or you become a member of an existing project. The second
can be done as follows:

\begin{enumerate}
\item First you need the name of the project you want to become a
  member of. You can ``search'' for the project (top middle side of the
  browser window) or you click on one of the ``Recently Registered
  Projects'' (if it is there).
\item Then you go to the project summary page (should be the default
  entry point). There is a window on the right side called
  ``Developer Info''. To join this project you need the permission of
  the project admin. So click on ``Request to join'' to send the
  project admin an email.
\item If the project admin decides to add you as developer you get an
  email which verifies your developer account. Now you have full SVN
  access (see section \ref{sec:scm} for details)
\end{enumerate}

\subsection{Registering a new project}
\label{sec:newproject}

Registering a new project is easy: Go to the Rforge website, login and
go to ``My Page'' section. You have a ``Register Project'' link in the
menu at the top of your page. Now insert the form and submit your
project.

\section{Source Code Management}
\label{sec:scm}

\subsection{How to get SCM to work}
\label{sec:scmhowto}
\textbf{Rforge} uses \textbf{Subversion}~\cite{subversion} for source code
management. Therefore you need a svn client (ie. ``tortoise svn'' on
windows machines~\cite{tortoisesvn}). For security reasons we use ssh tunneling for
developer accounts. This means you have to setup your machine
accordingly.

\subsubsection{Windows}
The software mentioned in this chapter can be found on~\cite{putty}.
\begin{enumerate}
\item First you need a ssh keypair: Generate and save the keys
  using ``puttygen.exe''.  
\item Upload the \textbf{public} key to \textbf{Rforge} using the webplatform: Go to
  ``My Page'' and then click on ``Account maintenance''. At the bottom
  of this page you click on ``edit keys'' in the ``Shell Account
  Information'' window (look at the hints on how to insert the key correctly).
\item Next you need an authentication agent like ``peagant.exe''. Load
  your \textbf{private} key with ``peagant.exe''.
\item Finally check out the repository using the URL given on the
  project website (tab ``SCM'') under ``developer account'' with
  ``Tortoise SVN'' or some other client.
\end{enumerate}

\subsubsection{Unix}

\begin{enumerate}
\item First you need a ssh keypair: Generate and save the keys
  using ``ssh-keygen'' on the command line.  
\item Upload the \textbf{public} key to \textbf{Rforge} using the
  webplatform: Go to
  ``My Page'' and then click on ``Account maintenance''. At the bottom
  of this page you click on ``edit keys'' in the ``Shell Account
  Information'' window (follow the instructions on how to insert the
  key correctly).
\item Finally check out the repository using the URL given on the
  project website (tab ``SCM'') under ``developer account'' with
  ``svn checkout''.
\end{enumerate}

\bibliographystyle{plainnat}
\bibliography{rforge-manual}

\end{document}