\documentclass[a4paper]{article}

% PREAMBLE
%\usepackage[austrian]{babel}
%\usepackage[utf8]{inputenc}
\usepackage[authoryear,round]{natbib}
\usepackage{graphicx}
\usepackage{hyperref}


\title{R-Forge user's manual}
\author{Stefan Theussl}
%\email{R-Forge@R-project.org}
%\institute{Department of Statistics and Mathematics\newline
%Vienna University of Economics and Business Administration}
\date{\today}

\begin{document}

\maketitle

%\tableofcontents

\section{Introduction}
\label{sec:intro}
For a decade the R Development Core Team has used collaborative
development tools like Subversion (SVN) provided by ETH Z\"urich. 
Many package
developers around the world use similar infrastructure as
collaborative development has become more and more important.
Now, the \textbf{R-project} wants to provide infrastructure not only
for the R-base developers but also for the entire R Community.

\textbf{R-Forge} (\url{http://R-Forge.R-project.org}) is the new
framework based on 
\textbf{GForge} which integrates various tools like
``SVN'' for collaborative work on source code or mailinglists,
etc. into one platform. It is a
place where all R developers and users can come together and exchange
their knowledge. The advantage of having the new meeting (or coding)
point is, that your package is going to be built and checked every day
with not only with the latest patched and release version but also the
newest devel version of R. You don't have to set up your own
build and check environment. That means you have your devel version of
your package on R-Forge and if you decide to release a new
version on CRAN, this can be done in a standardized way.

Furthermore, all package developers now have a
platform which allows presenting their work on the basis of project
websites or
news anouncements. Mailinglists or forums give developers convinient
instruments for communicating information to their partners or
interested people.
With the possibility to categorize your project in the project tree we
try to offer a way of searching packages not only by name but also by
topic. It is clear that these tree can't be complete, so people are
welcome to make suggestions on how we could improve it.
 
In this manual all relevant steps to get
started with \textbf{R-Forge} are going to be described. For a detailed
documentation regarding the \textbf{GForge} system itself I refer
to~\cite{manual:gforge}.\newline
\newline
The main features provided by \textbf{R-Forge} are:
\begin{description}
\item[SCM] means 'Source Code Management' and provides the most
  important feature namely ``SVN''.
\item[Daily package builds] can then be downloaded via\newline
  \url{install.packages("foo",url="R-Forge.R-project.org")}.
\end{description}
in addition there are other useful tools available
\begin{itemize}
\item The project tree offers you to assign your project to one or more
  topics (i.e.: finance, biostatistics, regression analysis, ...)
\item Forums are places to discuss certain topics with package
  developers.
\item Mailinglists can be managed by package admins.
\item News can be put on the front page if the R-Forge admins approve
  them.
\end{itemize}
Upcoming features:
\begin{itemize}
\item Bug tracking system
\item Wiki
\item Task management
\item Code snippets
\item Project help board
\end{itemize}

\section{Registration}
\label{sec:registration}

\subsection{Registering a  new user}

To register a new user, click on the ``New Account'' link on the top
right side of the browser window (\url{http://R-Forge.R-project.org}).
Fill out the form (there are descriptions and hints for each field on
this site) and click on ``Submit'' afterwards. You will receive an
email with
a URL to your specified email address. After clicking on this link
your account has been verified. Now you're able to login to the
website.
\newline

\textbf{Important note:} The tab ``My Page'' is the most important
page on \textbf{R-Forge}. There you can configure your account, you can see your
project memberships and see the items, which have been assigned to you
(i.e. bugs, feature requests, etc.).
\newline

Now you can start your own project (see section \ref{sec:newproject}
for details) or become a member of an existing project (section \ref{sec:joinproject}).

\subsection{Joining a project}
\label{sec:joinproject}
If you like to join an existing project you achieve this by doing the
following steps:
\begin{enumerate}
\item First you need the name of the project you want to become a
  member of. You can ``search'' for the project (top middle side of the
  browser window) or you click on one of the ``Recently Registered
  Projects'' (if it is there).
\item Then you go to the project summary page (should be the default
  entry point). There is a window on the right side called
  ``Developer Info''. To join this project you need the permission of
  the project admin. So click on ``Request to join'' to send the
  project admin an email.
\item If the project admin decides to add you as developer you will
  receive an email which verifies your developer account. Now you have
  full SVN access (see section \ref{sec:scm} for details).
\end{enumerate}

\subsection{Registering a new project}
\label{sec:newproject}

Registering a new project is easy: Go to the R-Forge website, login and
go to ``My Page'' section. You have a ``Register Project'' link in the
menu at the top of your page. Now insert the form and submit your
project. If the R-Forge admins accept the project you will be notified
by email and you will be able to start with your project on R-Forge.

\section{Source Code Management}
\label{sec:scm}

\subsection{How to get SCM to work}
\label{sec:scmhowto}

\textbf{R-Forge} uses
\textbf{Subversion}~(SVN, \url{http://subversion.tigirs.org}) for
source code management.
Therefore you need an SVN client (i.e. ``Tortoise SVN'' on
windows machines,~\url{http://tortoisesvn.tigris.org}). For security
reasons we use ssh tunneling\footnote{Secure shell (ssh) tunneling
  means that all network traffic is encrypted} for
developer accounts. This means you have to setup your machine
accordingly.

\subsubsection{Windows}

The software mentioned in this section can be found
on\newline
\url{http://www.chiark.greenend.org.uk/~sgtotham/putty/}.
\begin{enumerate}
\item First you need an ssh keypair: Generate and save the keys
  using \texttt{puttygen.exe}.   
\item Upload the \textbf{public} key to \textbf{R-Forge} using the webplatform: Go to
  ``My Page'' and then click on ``Account maintenance''. At the bottom
  of this page you click on ``edit keys'' in the ``Shell Account
  Information'' window. The key you enter here is typically of the
  form:
\begin{verbatim}
ssh-dsa AAAA\ldots foo@bar
\end{verbatim}
  The first field
  describes the type of key, the second field is the key itself, and
  the third field is a comment (it is important that there are no
  newlines within a key).
\item Next you need an authentication agent like \texttt{peagant.exe}. Load
  your \textbf{private} key with \texttt{peagant.exe}.
\item Finally check out the repository using the URL given on the
  project website (tab ``SCM'') under ``developer account'' with
  ``Tortoise SVN'' or some other client.
\end{enumerate}

\subsubsection{Unix}

\begin{enumerate}
\item First you need a ssh keypair: Generate and save the keys
  using \texttt{ssh-keygen} on the command line or use your existing keypair.  
\item Upload the \textbf{public} key to \textbf{R-Forge} using the
  webplatform: Go to
  ``My Page'' and then click on ``Account maintenance''. At the bottom
  of this page you click on ``edit keys'' in the ``Shell Account
  Information'' window. The key you enter here is typically of the
  form:
\begin{verbatim}
ssh-dsa AAAA\ldots foo@bar
\end{verbatim}
  The first field
  describes the type of key, the second field is the key itself, and
  the third field is a comment (it is important that there are no
  newlines within a key).
\item Finally check out the repository using the URL given on the
  project website (tab ``SCM'') under ``developer account'' with
  \texttt{svn checkout}.
\end{enumerate}

\subsection{Your project directory}
If you have checked out the repository of your project (see
section~\ref{sec:scmhowto} how to achieve this) for the first
time it contains two important pre-defined directories namely
\texttt{www} and \texttt{pkg}. They must not be
deleted otherwise R-Forge's core functionality will not be available.
(daily check and build of your package or project websites).
These two
directories are standardized and therefore are going to be described
in this section. The rest of your repository can be used as you like.

\subsubsection{/pkg directory}
Typically this directory contains the R package with the usual
\texttt{DESCRIPTION} and \texttt{R/}, \texttt{man/}, \texttt{data/}
directories etc (see \cite{Rcore:writing_R_extensions}
for more details).
In the future it will also be possible to have multiple
packages managed by a control file, however currently this feature is still
under development.

Furthermore, this directory will be checked out daily, the package is
checked and if it passes this procedure it is build and made available at\newline
\url{http://R-Forge.R-project.org/src/contrib/} (as source tar.gz and win32
.zip). It should be possible to install the package via
\texttt{install.packages("foo",url="R-Forge.R-project.org")} within R
then.

\subsubsection{/www directory}
This directory contains your project homepage which is available at
\texttt{http://<projectname>.R-Forge.R-project.org}.
Note that it will be checked out daily, so please take
into consideration that it will not be available right after you
commit your changes or updates. 



\bibliographystyle{plainnat}
\bibliography{rforge-manual}

\end{document}