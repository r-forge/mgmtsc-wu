
%E:\UNI\WS06\SVN Ordner\SVN neu\study\Rmetrics\CopulaFit\report

\section{Fit-Methoden}
\label{sec:fitmethoden}

Copulae Funktionen sind einfache Ausdr�cke von multivariaten Modellen,
jedoch kann darauf die klassische Statistik nicht angewandt
werden. Die einzige Theorie, die angewandt werden kann, ist die
asymptotische Maximum Likelihood Methode (MLE). Es gibt verschiedene
Methoden, die rechenintesive Berechnung der ML durchzuf�hren. Es wird
immer eine numerische Optimierungsfunktion ben�tigt,  welche die
Zielfunktion optimiert. In diesem Zusammenhang werden die in \emph{R}
implementierten Funktionen genutzt. Ein Problem bei der Optimierung
liegt darin, dass es viele Kombinationen von Randverteilungen und
Copulaes gibt. Es wird also die optimale L�sung der Kombination von
Copulae und Randverteilung suchen. 

Es wird immer von stetigen Zufallszahlen (Renditen) ausgegangen,
jedoch muss diese Bedingung nicht immer vorliegen.  

\subsection{Exact Maximum Likelihood (EML) method (one stage method)}
\label{sec:eml}

Bei der EML-Methode wird versucht, sowohl die Randverteilungen als
auch die Copula gleichzeitig �ber einen MLE zu optimieren. Hierbei
wird eine Datenmatrix $ \chi =
\left\{x_{1t},x_{2t},...,x_{nt}\right\}^{T}_{t=1} $ ben�tigt. Die zu
optimierende Funktion lautet dann: 

\begin{equation}
  \label{eq:EML}
  l(\theta) = \sum^{T}_{t=1} \ln c(F_1(x_{1t}),...,F_n(x_{nt})) + \sum^{T}_{t=1}\sum^{n}_{j=1} \ln  f_j(x_{jt})
\end{equation}


wobei $\theta$ alle Parameter der Randverteilungen und der Copula und
$c$ die Dichtefunktion der Copula darstellt. Nehmen wir nun eine
Vielzahl von Randverteilungen und Copulae her, so erh�lt man durch 

\begin{equation}
 \hat{\theta}_{MLE} = \max_{\theta \in \ominus} {l(\theta)}  
\end{equation}


den Maximum Likelihood Sch�tzer.

\subsection{Inference Functions for Margins (IFM) method (two stage method)}
\label{sec:ifm}

Da die EML bei h�heren Dimensionen sehr rechenintensiv werden kann,
verwendet man noch andere Methoden wie beispielsweise die
IFM-Methode. Betrachtet man die MLE-Methode genauer, so erkennt man,
dass diese aus zwei Teilen besteht, n�mlich einerseits aus dem
Sch�tzen der richtigen Copula und anderseits aus dem Sch�tzen der
besten Randverteilung. Genau diese Eigenschaft hat man sich bei der
IFM - Methode bedient. Hierbei werden zuerst die Randverteilungen
nach

\begin{equation}
  \label{eq:IFMrand}
  \hat{\theta}_{1} = \arg\max_{\theta_1} \sum^{T}_{t=1}\sum^{n}_{j=1} \ln {f_j(x_{jt};\theta_1)}  
\end{equation}


gesch�tzt. Im zweiten Schritt wird mit den angepassten
Randverteilungen die Copula nach 

\begin{equation}
  \label{eq:IFMcopula}
  \hat{\theta}_{2} = \arg\max_{\theta_2} \sum^{T}_{t=1} \ln {c(F_1(x_{1t}),...,F_n(x_{nt});\theta_1,\hat{\theta}_{1})}   
\end{equation}


gesch�tzt.

Der IFM-Sch�tzer ist somit definiert als Vektor $ \hat{\theta}_{IFM}=(\hat{\theta}_{1},\hat{\theta}_{2})$.

\subsection{Canonical Maximum Likelihood (CML) method}
\label{sec:cml}

Bei der vorhergehenden Methode wurde die ben�tigte Verteilungsfunktion
der Randverteilung gesch�tzt. Diesen Schritt kann man bei der
CML-Methode weglassen, indem die empirische Verteilungsfunktion
($\hat{F_i}(x_{it})$ mit $i = 1,....,n$) verwendet wird.

Mit der empirischen Verteilungsfunktion werden mittels MLE die
Parameter der zugrundeliegenden Copula gesch�tzt.

\begin{equation}
  \label{eq:CML}
  \hat{\theta}_{2} = \arg \max_{\theta_2} \sum^{T}_{t=1} \ln c(\hat{F_1}(x_{1t}),...,\hat{F_n}(x_{nt});\theta_1,\hat{\theta}_{2})  
\end{equation}
