
%E:\UNI\WS06\SVN Ordner\SVN neu\study\Rmetrics\CopulaFit\report

\section{Fit-Methoden}
\label{sec:fitmethoden}

Copulae Funktionen sind einfache Ausdr�cke von multivariaten Modellen jedoch kann darauf die klassische Statistik nicht angewandt werden. Die einzige Theorie die angewandt werden kann ist die asymptotische Maximum Likelihood Methode (MLE). Weiters gibt auch noch andere Methoden die rechenintesive Berechnung der MLE�s durchzuf�hren. F�r unsere Berechnung ben�tigen wir immer eine numerische Optimierungsfunktion die uns unsere Zielfunktion optimiert. In diesem Zusammenhang verwenden wir die in \emph{R} implementierten Funktionen. Ein Problem bei der Optimierung liegt darin, dass es viele Kombinantionen von Randverteilungen und Copulaes gibt und das wir die optimale L�sung der Kombination von Copulae und Randverteilung suchen.

Es wird immer von stetigen Zufallszahlen (Renditen) ausgegangen, jedoch muss diese Bedingung nicht immer vorliegen. 

\subsection{Exact Maximum Likelihood (EML) method (one stage method)}
\label{sec:eml}

Hierbei wird versucht sowohl die Randverteilungen als auch die Copulae gleichzeitig �ber einen MLE zu optimieren.
Wobei wir eine Datenmatrix $ \chi = \left\{x_{1t},x_{2t},...,x_{nt}\right\}^{T}_{t=1} $ vorhanden ist. Die zu optimierende Funktion lauted dann:

$$ l(\theta) = \sum^{T}_{t=1} ln c(F_1(x_{1t}),...,F_n(x_{nt})) + \sum^{T}_{t=1}\sum^{n}_{j=1}ln f_j(x_{jt})$$

wobei $\theta$ alle Parameter der Randverteilung und er Copulae darstellt. Nehmen wir nun eine Vielzahl von Randverteilungen und Copulae her so erhalten durch:

$$ \hat{\theta}_{MLE} = max_{\theta \in \ominus} l(\theta) $$

den Maximum Likelihood Sch�tzer.

\subsection{Inference Functions for Margins (IFM) method (two stage method)}
\label{sec:ifm}

Da die EML bei h�heren Dimensionen sehr Rechenintensiv werden kann werden noch andere Methoden wie die IFM verwendet. Wenn wir uns die obigen MLE anschauen so sehen wir, dass dieser aus 2 Funktionen besteht. Einerseits aus dem Sch�tzen der richtigen Copulae und anderseits dem sch�tzen der besten Randverteilung. Genau diese Eigenschaft hat man sich bei der IFM - Methode bedient. Hierbei werden zuerst die Randverteilungen nach,

$$ \hat{\theta}_{1} = ArgMax_{\theta_1} \sum^{T}_{t=1}\sum^{n}_{j=1}ln f_j(x_{jt};\theta_1) $$

gesch�tzt. Als zweiten Schritt wird mit den gefitteten Randverteilungen die Copulae nach,

$$ $$ \hat{\theta}_{2} = ArgMax_{\theta_2} \sum^{T}_{t=1} ln c(F_1(x_{1t}),...,F_n(x_{nt});\theta_1,\hat{\theta}_{1}) $$

gesch�tzt.

Der IFM Sch�tzer ist somit definiert als Vektor $ \hat{\theta}_{IFM}=(\hat{\theta}_{1},\hat{\theta}_{2})�$.

\subsection{Canonical Maximum Likelihood (CML) method}
\label{sec:cml}

Bei vorhergehenden Methode haben wir ben�tigte Verteilungsfunktion der Randverteilungen gesch�tzt. Diesen Schritt wollen wir uns mit der CML-Methode ersparen. Dazu ben�tigen wir aber die Empirische Verteilungsfunktion ($\hat{F_i}(x_{it}) mit i = 1,....,n$) die wie folgt definiert ist:

\begin{Schunk}
\begin{Sinput}
> pemp <- function(t, x) {
+     F <- NULL
+     for (i in 1:length(t)) {
+         F <- c(F, 1/length(x) * sum(x <= t[i]))
+     }
+     F
+ }
\end{Sinput}
\end{Schunk}

Mit der empirischen Verteilungsfunktion berechnen wir Mittels MLE die Parameter der zugrundeliegenden Copulae.

$$ $$ \hat{\theta}_{2} = ArgMax_{\theta_2} \sum^{T}_{t=1} ln c(\hat{F_}1(x_{1t}),...,\hat{F_n}(x_{nt});\theta_1,\hat{\theta}_{2}) $$





