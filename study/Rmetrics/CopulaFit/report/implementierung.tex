
\section{Implementierung in R}
\label{sec:R}

Die Fit-Funktion ist als generische Funktion implementiert und sieht
folgenderma�en aus:

\begin{Schunk}
\begin{Sinput}
> CopulaFit <- function(x, y, method = "CML", returns = FALSE, ...) {
+     UseMethod("CopulaFit")
+ }
\end{Sinput}
\end{Schunk}

\begin{itemize}
\item \verb|x| und \verb|y| sind die Datenpunkte der zwei Zeitreihen,
  die untersucht werden sollen. Es m\"ussen entweder Kurswerte oder
  gleich Renditen eingegeben werden.

\item \verb|method| steht hierbei f\"ur die Fit-Methode, die verwendet 
werden soll. Es stehen die Methoden \verb|"CML"|, \verb|"EML"| und 
\verb|"IFM"| zur Verf\"ugung.

\item \verb|returns| ist eine Kontrollvariable. Sind in den
  Datenvektoren \verb|x| und \verb|y| bereits Renditen enthalten, so
  muss \verb|returns = TRUE| gesetzt werden.
\end{itemize}

Die ben�tigte Klasse f\"ur die Funktion \verb|CopulaFit| hei�t
\verb|"ml"|. Die Datenreihen werden jedoch automatisch in diese Klasse
umgewandelt. Allerdings kann dies auch manuell mit der Funktion geschehen:

\begin{Schunk}
\begin{Sinput}
> as.ml <- function(x, returns = FALSE) {
+     library(fSeries)
+     if (returns == FALSE) {
+         x <- as.timeSeries(x)
+         x <- returnSeries(x)
+         x <- x@Data
+     }
+     class(x) <- "ml"
+     x
+ }
\end{Sinput}
\end{Schunk}

Je nachdem, welche \verb|method| in \verb|CopulaFit| angegeben wird,
wird die passende Funktion ausgef\"uhrt:

\subsection{EML-Methode}
\label{sec:Reml}

Die Funktion, die aufgerufen wird, hei�t \verb|.EMLfit|. Die Parameter
werden automatisch \"ubergeben. Diese Methode funktioniert derzeit
f\"ur zw�lf Copulae (archimedische Copulae 1, 3-6, 9-10,
12-14, 16-17 nach \cite{nelsen2006}) und f\"ur alle vier Kombinationen
von Randverteilungen 
(zB Randverteilung von \verb|x| = t-Verteilung und Randverteilung von
\verb|y| = Normalverteilung). Bei Anwendung anderer Verteilungen
treten derzeit noch Fehler auf.

Da die Parameter der Randverteilungen und der Copulae gleichzeitig
gesch�tzt werden m\"ussen, gibt es einige Hindernisse, die bei dieser
EML-Methode \"uberwunden werden m\"ussen: 

\begin{itemize}
\item \verb|upperbounds| und \verb|lowerbounds| beschreibt die oberen
  und unteren Grenzen, welche w�hrend der Parametersch�tzung nicht
  unter- bzw. \"uberschritten werden d\"urfen.
\item \verb|startwerte| \"ubergibt dem Optimierer passende Startwerte.
\item In \verb|fun| wird die zu minimierende Zielfunktion
  definiert. Diese entspricht Gleichung \ref{eq:EML}. Mit der Funktion
  \verb|do.call| wird daf\"ur gesorgt, dass immer alle m�glichen
  Kombinationen von Copulae und Randverteilungen mit den passenden
  Parametern aufgerufen werden. Die m�glichen Kombinationen werden mit
  \verb|makelist()| erzeugt.
\end{itemize}


\subsection{IFM-Methode}
\label{sec:Rifm}

Diese Methode funktioniert derzeit ebenfalls f\"ur zw�lf Copulae
(archimedische Copulae 1, 3-6, 9-10, 12-14, 16-17 nach
\cite{nelsen2006}) und f\"ur jeweils
f\"unf verschiedene Randverteilungen (Normal- und t-Verteilung,
Generalised Hyperbolic Distribution, Hyperbolic Distribution, Normal
Inverse Gaussian Distribution).

Die Funktion \verb|.IFMfit| sucht im ersten Schritt nach den
Randverteilungen und deren Parametern, die am besten zu den
Datenreihen \verb|x| und \verb|y| passen. Dies geschieht durch
internen Aufruf der Funktion \verb|.marginFit|. \verb|.pmarginFit|
berechnet dann die Wahrscheinlichkeiten der Datenpunkte mit den
angepassten Verteilungen.

In \verb|fun| wird wieder die zu minimierende Zielfunktion
definiert. Diese entspricht Gleichung \ref{eq:IFMcopula}. Da bei der
Sch�tzung der Copula nur der Parameter f\"ur die jeweilige Copula
angepasst wird, reicht es einen Startparameter und die Grenzen f\"ur
jede Copula zu definieren. Dies geschieht in den Objekten \verb|alpha|
und \verb|range|.



\subsection{CML-Methode}
\label{sec:Rcml}

Hier m\"ussen zun�chst die empirischen Verteilungsfunktionen der
Datenreihen \verb|x| und \verb|x| berechnet werden. Dazu wurde
folgende Funktion implementiert:

\begin{Schunk}
\begin{Sinput}
> pemp <- function(t, x) {
+     F <- NULL
+     for (i in 1:length(t)) {
+         F <- c(F, 1/length(x) * sum(x <= t[i]))
+     }
+     F
+ }
\end{Sinput}
\end{Schunk}

Somit m\"ussen auch hier nur die Parameter der Copulae gesch�tzt
werden. Die zu optimierende Funktion (Gleichung \ref{eq:CML}) ist in \verb|fun|
definiert. Startparameter und Grenzen werden in \verb|alpha| und
\verb|range| angegeben.


\subsection{Anwendungsbeispiel}
\label{sec:Rbeispiel}

Nun soll die Funktion \verb|CopulaFit| an einem genierierten
Beispieldatensatz angewendet werden:


\begin{Schunk}
\begin{Sinput}
> x <- rnorm(100, 760, 23)
> y <- rnorm(100, 500, 99)
> x <- as.ml(x, returns = FALSE)
> y <- as.ml(y, returns = FALSE)
\end{Sinput}
\end{Schunk}

Die Datenvektoren werden jetzt aufbereitet. Auch diese haben eine
eigene print-Methode:

\begin{Schunk}
\begin{Sinput}
> x
\end{Sinput}
\begin{Soutput}
This is a time series containing  99 returns.
Prepared for calling the function "CopulaFit" 
\end{Soutput}
\begin{Sinput}
> y
\end{Sinput}
\begin{Soutput}
This is a time series containing  99 returns.
Prepared for calling the function "CopulaFit" 
\end{Soutput}
\end{Schunk}

Nun kann die Anpassung durchgef\"uhrt werden. Im folgenden Beispiel
werden die Copulae der oben definierten Datenreihen nach der
IFM-Methode gesch�tzt:

\begin{Schunk}
\begin{Sinput}
> fit <- CopulaFit(x, y, method = "IFM", returns = TRUE)
\end{Sinput}
\end{Schunk}

Die Ausgabe von \verb|fit| liefert nun eine \"Ubersicht \"uber die
Parameter, den Log-Likelihood-Wert und der verwendeten Methode aller
gefitteten Copulae.

\begin{Schunk}
\begin{Sinput}
> fit
\end{Sinput}
\begin{Soutput}
            fit.family      fit.par fit.objective fit.method
1   Archimedian Type 1 7.752083e+00   -0.81519657        IFM
2   Archimedian Type 3 1.000000e+00   -0.15449225        IFM
3   Archimedian Type 4 5.707346e+00   -0.84670550        IFM
4   Archimedian Type 5 1.682922e+01   -0.84271404        IFM
5   Archimedian Type 6 8.518132e+00   -0.81247049        IFM
6   Archimedian Type 9 1.000000e+00   -0.13150519        IFM
7  Archimedian Type 10 7.458241e-01   -0.02627407        IFM
8  Archimedian Type 12 4.009337e+00   -0.85369003        IFM
9  Archimedian Type 13 1.371508e+01   -0.84091227        IFM
10 Archimedian Type 14 5.306909e+00   -0.84966595        IFM
11 Archimedian Type 16 2.422727e+09   -0.15449225        IFM
12 Archimedian Type 17 2.477535e+01   -0.83985273        IFM
\end{Soutput}
\end{Schunk}

Um schneller jene Copula w�hlen zu k�nnen, welche den maximalen
Log-Likelihood-Wert liefert, kann der Befehl \verb|summary| verwendet werden:

\begin{Schunk}
\begin{Sinput}
> summary(fit)
\end{Sinput}
\begin{Soutput}
 Die best-fit Copula ist Typ:  Archimedian Type 12 
 Minimaler Zielfunktionswert:  -0.85369 

Summary:
        a.family.ind. a.objective.ind. a.method.ind. a.par.ind.
1 Archimedian Type 12         -0.85369           IFM   4.009337
\end{Soutput}
\end{Schunk}
