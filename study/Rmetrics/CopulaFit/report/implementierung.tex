
\section{Implementierung in R}
\label{sec:R}

Die Fit-Funktion ist als generische Funktion implementiert und sieht
folgenderma�en aus:

\begin{Schunk}
\begin{Sinput}
> CopulaFit <- function(x, y, method = "CML", returns = FALSE, 
+     ...) {
+     UseMethod("CopulaFit")
+ }
\end{Sinput}
\end{Schunk}

\begin{itemize}
\item \verb|x| und \verb|y| sind die Datenpunkte der zwei Zeitreihen,
  die untersucht werden sollen. Es m\"ussen entweder Kurswerte oder
  gleich Renditen eingegeben werden.

\item \verb|method| steht hierbei f\"ur die Fit-Methode, die verwendet 
werden soll. Es stehen die Methoden \verb|"CML"|, \verb|"EML"| und 
\verb|"IFM"| zur Verf\"ugung.

\item \verb|returns| ist eine Kontrollvariable. Sind in den
  Datenvektoren \verb|x| und \verb|y| bereits Renditen enthalten, so
  muss \verb|returns = TRUE| gesetzt werden.
\end{itemize}

Die ben�tigte Klasse f\"ur die Funktion \verb|CopulaFit| hei�t
\verb|"ml"|. Die Datenreihen werden jedoch automatisch in diese Klasse
umgewandelt. Allerdings kann dies auch manuell mit der Funktion geschehen:

\begin{Schunk}
\begin{Sinput}
> as.ml <- function(x, returns = FALSE) {
+     library(fSeries)
+     if (returns == FALSE) {
+         x <- as.timeSeries(x)
+         x <- returnSeries(x)
+         x <- x@Data
+     }
+     class(x) <- "ml"
+     x
+ }
\end{Sinput}
\end{Schunk}

Je nachdem, welche \verb|method| in \verb|CopulaFit| angegeben wird,
wird die passende Funktion ausgef\"uhrt:

\subsection{EML-Methode}
\label{sec:Reml}

Die Funktion, die aufgerufen wird, hei�t \verb|.EMLfit|. Die Parameter
werden automatisch \"ubergeben. Diese Methode funktioniert derzeit
f\"ur zw�lf Copulae (archimedische Copulae 1, 3-6, 9-10,
12-14, 16-17 nach \cite{nelsen2006}) und f\"ur alle vier Kombinationen
von Randverteilungen 
(zB Randverteilung von \verb|x| = t-Verteilung und Randverteilung von
\verb|y| = Normalverteilung). Bei Anwendung anderer Verteilungen
treten derzeit noch Fehler auf.

Da die Parameter der Randverteilungen und der Copulae gleichzeitig
gesch�tzt werden m\"ussen, gibt es einige Hindernisse, die bei dieser
EML-Methode \"uberwunden werden m\"ussen: 

\begin{itemize}
\item \verb|upperbounds| und \verb|lowerbounds| beschreibt die oberen
  und unteren Grenzen, welche w�hrend der Parametersch�tzung nicht
  unter- bzw. \"uberschritten werden d\"urfen.
\item \verb|startwerte| \"ubergibt dem Optimierer passende Startwerte.
\item In \verb|fun| wird die zu minimierende Zielfunktion
  definiert. Diese entspricht Gleichung \ref{eq:EML}. Mit der Funktion
  \verb|do.call| wird daf\"ur gesorgt, dass immer alle m�glichen
  Kombinationen von Copulae und Randverteilungen mit den passenden
  Parametern aufgerufen werden. Die m�glichen Kombinationen werden mit
  \verb|makelist()| erzeugt.
\end{itemize}


\subsection{IFM-Methode}
\label{sec:Rifm}

Diese Methode funktioniert derzeit ebenfalls f\"ur zw�lf Copulae
(archimedische Copulae 1, 3-6, 9-10, 12-14, 16-17 nach
\cite{nelsen2006}) und f\"ur jeweils
f\"unf verschiedene Randverteilungen (Normal- und t-Verteilung,
Generalised Hyperbolic Distribution, Hyperbolic Distribution, Normal
Inverse Gaussian Distribution).

Die Funktion \verb|.IFMfit| sucht im ersten Schritt nach den
Randverteilungen und deren Parametern, die am besten zu den
Datenreihen \verb|x| und \verb|y| passen. Dies geschieht durch
internen Aufruf der Funktion \verb|.marginFit|. \verb|.pmarginFit|
berechnet dann die Wahrscheinlichkeiten der Datenpunkte mit den
angepassten Verteilungen.

In \verb|fun| wird wieder die zu minimierende Zielfunktion
definiert. Diese entspricht Gleichung \ref{eq:IFMcopula}. Da bei der
Sch�tzung der Copula nur der Parameter f\"ur die jeweilige Copula
angepasst wird, reicht es einen Startparameter und die Grenzen f\"ur
jede Copula zu definieren. Dies geschieht in den Objekten \verb|alpha|
und \verb|range|.



\subsection{CML-Methode}
\label{sec:Rcml}

Hier m\"ussen zun�chst die empirischen Verteilungsfunktionen der
Datenreihen \verb|x| und \verb|x| berechnet werden. Dazu wurde
folgende Funktion implementiert:

\begin{Schunk}
\begin{Sinput}
> pemp <- function(t, x) {
+     F <- NULL
+     for (i in 1:length(t)) {
+         F <- c(F, 1/length(x) * sum(x <= t[i]))
+     }
+     F
+ }
\end{Sinput}
\end{Schunk}

Somit m\"ussen auch hier nur die Parameter der Copulae gesch�tzt
werden. Die zu optimierende Funktion (Gleichung \ref{eq:CML}) ist in \verb|fun|
definiert. Startparameter und Grenzen werden in \verb|alpha| und
\verb|range| angegeben.


\subsection{Anwendungsbeispiel}
\label{sec:Rbeispiel}

\subsubsection{Durchf\"uhrung einer Anpassung}
\label{sec:Durchfuehrung}

Nun soll die Funktion \verb|CopulaFit| an einem genierierten
Beispieldatensatz angewendet werden:


\begin{Schunk}
\begin{Sinput}
> x <- rnorm(100, 760, 23)
> y <- rnorm(100, 500, 99)
> x <- as.ml(x, returns = FALSE)
> y <- as.ml(y, returns = FALSE)
\end{Sinput}
\end{Schunk}

Die Datenvektoren werden jetzt aufbereitet. Auch diese haben eine
eigene print-Methode:

\begin{Schunk}
\begin{Sinput}
> x
\end{Sinput}
\begin{Soutput}
This is a time series containing  99 returns.
Prepared for calling the function "CopulaFit" 
\end{Soutput}
\begin{Sinput}
> y
\end{Sinput}
\begin{Soutput}
This is a time series containing  99 returns.
Prepared for calling the function "CopulaFit" 
\end{Soutput}
\end{Schunk}

Nun kann die Anpassung durchgef\"uhrt werden. Im folgenden Beispiel
werden die Copulae der oben definierten Datenreihen nach der
IFM-Methode gesch�tzt:

\begin{Schunk}
\begin{Sinput}
> ifm <- CopulaFit(x, y, method = "IFM", returns = TRUE)
\end{Sinput}
\end{Schunk}

Die Ausgabe von \verb|fit| liefert nun eine \"Ubersicht \"uber die
Parameter, den Log-Likelihood-Wert und der verwendeten Methode aller
gefitteten Copulae.

\begin{Schunk}
\begin{Sinput}
> ifm
\end{Sinput}
\begin{Soutput}
   fit.family     fit.par fit.objective fit.method fit.convergence  fit.AIC
1    Archm. 1  0.20053613 -2.386252e-03        IFM               0 1.995227
2    Archm. 3  0.03803663 -1.869869e-05        IFM               0 1.999963
3    Archm. 4  1.00000000 -2.242875e-18        IFM               0 2.000000
4    Archm. 5  0.06693089 -1.636559e-05        IFM               0 1.999967
5    Archm. 6  1.00000000  1.233581e-17        IFM               0 2.000000
6    Archm. 9  0.00000000  0.000000e+00        IFM               0 2.000000
7   Archm. 10  0.00000000  0.000000e+00        IFM               0 2.000000
8   Archm. 12  1.00000000  5.460098e-02        IFM               0 2.109202
9   Archm. 13  1.17860348 -4.582247e-04        IFM               0 1.999084
10  Archm. 14  1.00000000  5.460098e-02        IFM               0 2.109202
11  Archm. 16  0.32253662  1.949495e-02        IFM               0 2.038990
12  Archm. 17 -0.70131228 -9.545983e-05        IFM               0 1.999809
\end{Soutput}
\end{Schunk}

Um schneller jene Copula w�hlen zu k�nnen, welche den maximalen
Log-Likelihood-Wert liefert, kann der Befehl \verb|summary| verwendet
werden. Die Ausgabe zeigt die besten drei Copulae an:

\begin{Schunk}
\begin{Sinput}
> summary(ifm)
\end{Sinput}
\begin{Soutput}
 Die best-fit Copula ist Typ:  Archm. 1 
 Minimaler Zielfunktionswert:  -0.002386252 

Summary:
  a.family.ind. a.objective.ind. a.method.ind. a.par.ind. a.AIC.ind.
1      Archm. 1    -2.386252e-03           IFM  0.2005361   1.995227
2     Archm. 13    -4.582247e-04           IFM  1.1786035   1.999084
3     Archm. 17    -9.545983e-05           IFM -0.7013123   1.999809
  a.convergence.ind.
1                  0
2                  0
3                  0
\end{Soutput}
\end{Schunk}

Weitere Informationen k�nnen \"uber S3-Attribute ausgegeben werden. Es
stehen folgende Attribute zur Verf\"ugung:

\begin{Schunk}
\begin{Sinput}
> attributes(ifm)
\end{Sinput}
\begin{Soutput}
$names
 [1] "margin1"      "distrmargin1" "margin2"      "distrmargin2" "family"      
 [6] "par"          "objective"    "AIC"          "convergence"  "message"     
[11] "iterations"   "method"      

$class
[1] "mloutput"
\end{Soutput}
\end{Schunk}

Mit Hilfe dieser Attribute k�nnen zus�tzlich noch Informationen \"uber
die Randverteilungen, die Anzahl der Iterationen und eine Nachricht
\"uber das Konvergenzverhalten ausgegeben werden.


\subsubsection{Vergleich der Fit-Methoden}
\label{sec:VergleichFitMethoden}

Nun sollen die verschiednen Fit-Methoden verglichen werden. Dazu
werden zwei Zeitreihen verwendet, die von \emph{YahooFinance} besorgt
wurden. Es handelt sich um Zeitreihen von ATX- und NIKKEI-Werten, die
schon mit \verb|as.ml| f\"ur die Anpassung vorbereitet worden sind.

\begin{Schunk}
\begin{Sinput}
> ifm <- CopulaFit(x, y, method = "IFM", returns = TRUE)
> cml <- CopulaFit(x, y, method = "CML", returns = TRUE)
> eml <- CopulaFit(x, y, method = "EML", returns = TRUE)
\end{Sinput}
\end{Schunk}

\begin{Schunk}
\begin{Sinput}
> summary(cml)
\end{Sinput}
\begin{Soutput}
 Die best-fit Copula ist Typ:  Archm. 5 
 Minimaler Zielfunktionswert:  -0.0001179483 

Summary:
  a.family.ind. a.objective.ind. a.method.ind.  a.par.ind. a.AIC.ind.
1      Archm. 5    -0.0001179483           CML  0.09181627   1.999764
2      Archm. 3    -0.0001178839           CML  0.04533770   1.999764
3     Archm. 17    -0.0001069186           CML -0.84801622   1.999786
  a.convergence.ind.
1                  0
2                  0
3                  0
\end{Soutput}
\begin{Sinput}
> summary(ifm)
\end{Sinput}
\begin{Soutput}
 Die best-fit Copula ist Typ:  Archm. 5 
 Minimaler Zielfunktionswert:  -0.0001063884 

Summary:
  a.family.ind. a.objective.ind. a.method.ind.   a.par.ind. a.AIC.ind.
1      Archm. 5    -1.063884e-04           IFM  0.102855182   1.999787
2      Archm. 3    -1.053455e-04           IFM  0.050244838   1.999789
3      Archm. 1    -9.037212e-05           IFM -0.008746649   1.999819
  a.convergence.ind.
1                  0
2                  0
3                  0
\end{Soutput}
\begin{Sinput}
> summary(eml)
\end{Sinput}
\begin{Soutput}
 Die best-fit Copula ist Typ:  Archm. 5 
 Minimaler Zielfunktionswert:  -2.299921 

Summary:
  a.family.ind. a.objective.ind. a.method.ind. a.parCopula.ind. a.AIC.ind.
1      Archm. 5        -2.299921           EML        12.529208   3.400158
2      Archm. 4        -2.294417           EML         6.753759   3.411167
3     Archm. 14        -2.293418           EML         6.870347   3.413164
  a.convergence.ind.
1                  1
2                  1
3                  1
\end{Soutput}
\end{Schunk}

Wie man sieht, empfiehlt jede Fit-Methode die Archimedische Copula 5,
wobei die Konvergenz bei der EML-Methode nicht gegeben ist. Um
genauere Informationen \"uber die Konvergenzvariable zu erhalten, sei
auf die Hilfe von \verb|nlminb| verwiesen.

Bei der IFM- und CML-Methode sind die Parameter sehr �hnlich, nur die
EML-Methode liefert einen unbrauchbaren Wert f\"ur $\theta$. Auch die
zweitbesten Anpassungen stimmen bei CML- und IFM-Methode \"uberein.

Wie erwartet, ist die Rechenzeit bei der EML-Methode am
l�ngsten. Jedoch kann trotz der hohen Anzahl an Iterationen keine
Konvergenz erreicht werden.
