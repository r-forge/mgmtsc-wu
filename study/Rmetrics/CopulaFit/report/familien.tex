
\section{Familien von Copulae}
\label{sec:familien}

Aus obiger Darstellung geht klar hervor, dass es viele Funktionen
gibt, welche die Eigenschaften und Voraussetzungen einer Copula
erf\"ullen. F\"ur statistische und �konometrische Zwecke ist es daher
durchaus dienlich, eine Reihe von verschiedenen Copulae zur
Verf\"ugung zu haben. Man unterscheidet vor allem zwei gro�e Gruppen
von Copula-Familien, n�mlich die archimedischen Copulae und die
elliptischen Copulae.


\subsection{Elliptische Copulae}
\label{sec:elliptische}

Mit einfachen Worten gesagt, sind ellipitische Copulae die Copulae von
elliptischen Verteilungen. Daher ist es nicht verwunderlich, dass die
\emph{Gau�'sche Copula} (oder Normal Copula) und die \emph{t-Copula}
die wichtigsten Vertreter dieser Familie sind.

Elliptische Copulae bieten den Vorteil, dass sie viele Eigenschaften
einer multivariaten Normalveteilung mit sich bringen, dass sie leicht
berechenbar sind und dass Simulationen von elliptischen Verteilungen
(und somit auch von elliptischen Copulae) relativ einfach ist.

\subsubsection{Normal-Copula}
\label{sec:normal}

Die Gau�'sche Copula (Normal-Copula) ist die Copula der multivariaten
Normalverteilung. Sei $\mathbf{u}$ ein Vektor mit $n$ normalverteilten
Werten $\mathbf{x}$, so dass $(u_1, \ldots, u_n) = (F_1(x_1), \ldots,
F_n(x_n))$ wobei $F_i$ die Verteilungsfunktion einer Normalverteilung
darstellt. Dann ist die Gau�'sche Copula definiert mit

\begin{equation}
  \label{eq:gauss}
  C_G(\mathbf{u}) = \Phi_n(\Phi^{-1}(u_1), \ldots, \Phi^{-1}(u_n))
\end{equation}

wobei $\Phi_n$ die gemeinsame Verteilungsfunktion einer $n$-dimensionalen
Standardnormalverteilung mit Korrelationsmatrix $R$ ist und
$\Phi^{-1}$ die Quantilsfunktion der univariaten
Standardnormalverteilung ist.

Im zweidimensionalen Fall reduziert sich die Copula auf

\begin{equation}
  \label{eq:gauss2dim}
  C_G(u,v) = \int_{-\infty}^{\Phi^{-1}(u)}
  \int_{-\infty}^{\Phi^{-1}(v)} \frac{1}{2 \pi \sqrt{(1 - \rho^2)}}
  \exp{ \left( -\frac{s^2 - 2 \rho s t + t^2} {2 (1 - \rho^2)}
    \right) }ds dt
\end{equation}

wobei $\rho$ dem Korrelationskoeffizienten nach Pearson entspricht.


\subsubsection{t-Copula}
\label{sec:tcopula}

Die t-Copula bezieht sich nun auf die multivariate t-Verteilung. Die
R�nder sind also univariat t-verteilt, dh: $(u_1, \ldots, u_n) =
t_{\nu}(x_1, \ldots, x_n)$. Sei $T_{\nu}^n$ die standardisierte
$n$-dimensionale t-Verteilung und $t_{\nu}^{-1}$ die univariate
Quantilsfunktion, so ist die t-Copula $C_t$ definiert mit

\begin{equation}
  \label{eq:t}
  C_t(\mathbf{u}) = T_{\nu} (t_{\nu}^{-1}(u_1), \ldots, t_{\nu}^{-1}(u_n))
\end{equation}

Im zweidimensionalen Fall mit Korrelationskoeffizienten $\rho$
vereinfacht sich das ganze auf 

\begin{equation}
  \label{eq:t2dim}
  C_t(u, v) = \int_{-\infty}^{t_{\nu}^{-1}(u)}
  \int_{-\infty}^{t_{\nu}^{-1}(v)} \frac{1}{2 \pi \sqrt{(1 - \rho^2)}}
  \exp{ \left( -\frac{s^2 - 2 \rho s t + t^2} {2 (1 - \rho^2)}
    \right) }ds dt
\end{equation}


\subsection{Archimedische Copulae}
\label{sec:archimedische}

Eine weitere wichtige Familie von Copulae sind die archimedischen
Copulae. Diese werden nach \cite{nelsen2006} vor allem aus folgenden
Gr\"unden gerne angewendet:

\begin{itemize}
\item Sie k�nnen sehr einfach erzeugt werden.
\item Es gibt viele parametrische Familien von Copulae, die ebenfalls
  zu den archimedischen Copulae geh�ren.
\item Archimedische Copulae besitzen viele gew\"unschte Eigenschaften.
\end{itemize}

So vielf�ltig diese Klasse ist, so unterschiedlich ist auch die
Abh�ngigkeitsstruktur, die erzeugt werden
kann. Auf die Darstellung von Copulae dieser Familie wird hier
verzichtet. Dazu sei auf \cite{nelsen2006} S. 116 -- 119 verwiesen, wo
eine \"Ubersicht \"uber 22 verschiedene Copulae (jeweils mit einem
Parameter) gegeben wird. H�ufig verwendet werden von diesen vor allem
die \emph{Clayton Copula} und die \emph{Gumbel Copula} (Copula 1 und 4
bei \cite{nelsen2006}).

An dieser Stelle sei nur noch die Einfachheit der Konstruktion
demonstriert\footnote{vgl. \cite{fantazzini2003}}: Sei $\varphi$ 
eine stetige und konvexe Funktion von $\mathbf{I}$ nach $[0, \infty]$,
sodass $\varphi(1) = 0$ und $\varphi(0) = \infty$. Sei $\varphi^{[-1]}$ die
Pseudo-Inverse von $\varphi$, f\"ur die gilt:

\begin{displaymath}
  \varphi^{[-1]}(t) = \left\{ \begin{array}{ll}
    \varphi^{-1}(t) & \textrm{falls $0 \leq t \leq \varphi(0)$}\\
    0 & \textrm{falls $\varphi(0) \leq t \leq \infty$}
  \end{array} \right.
\end{displaymath}

Erf\"ullt $\varphi$ die Bedingung der Konvexit�t, so gilt nun f\"ur
den zweidimensionalen Fall

\begin{equation}
  \label{eq:archm}
  C(u, v) = \varphi^{[-1]}(\varphi(u) + \varphi(v))
\end{equation}

Daher wird $\varphi$ auch als Generatorfunktion bezeichnet. F\"ur die
Erweiterung auf den multivariaten Fall sei auf \cite{embrechts2003} verwiesen.
