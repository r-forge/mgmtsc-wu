
\section{Einleitung}
\label{sec:einleitung}

Es besteht das Problem, dass die Finanzm�rkte der Realit�t h�ufig
nicht den Annahmen der Finanzmarkttheorie entsprechen. So sind
beispielsweise Renditen von Marktdaten h�ufig unkorreliert, aber
dennoch abh�ngig, sie weisen sogenannte \emph{fat tails} auf und
entsprechen somit nicht der Annahme der Normalverteilung, treten in
Clustern auf und die Volatilit�t ist zuf�llig. 

Somit besteht vor allem die Schwierigkeit, die Verteilungsfunktion
eines Portfolios zu sch�tzen. Man kann zwar (unter Aufgabe der
\emph{``Normalit�t''}  sehr leicht die Randverteilungen der
einzelnen Instrumente des Portfolios sch�tzen, allerdings ist es
schwierig die gemeinsame Verteilung aller Instrumente abzusch�tzen.

Die Unbekanntheit der gemeinsamen Verteilungsfunktion f\"uhrt zu zwei
gro�en Problemen: 

\begin{itemize}
\item Zun�chst ist es f\"ur das Risikomanagement, das h�ufig auf Basis
  des \emph{VaR (Value at Risk)} betrieben wird, sehr schwierig,
  VaR-Werte zu finden, die tats�chlich jenen der gemeinsamen
  Verteilung entsprechen. Angenommen man betrachtet zwei Zeitreihen
  zweier unterschiedlicher Renditen. Kann man nun, um den VaR der
  gemeinsamen Verteilung zu sch�tzen, den VaR der beiden einzelnen
  Zeitreihen einfach addieren bzw. unter welchen Bedingungen ist dies
  die Obergrenze f\"ur den VaR\footnote{vgle \cite{embrechts2003}}.

\item Au�erdem ist die Sch�tzung der Abh�ngigkeitsstruktur innerhalb
  der Instrumente eines Portfolios sehr schwierig. Die Korrelation ist
  eine gute Ma�zahl f\"ur multivariate Normalverteilungen, weist aber
  nach \cite{fantazzini2003} drei wesentliche Nachteile auf:
  
  \begin{enumerate}
  
  \item Die Varianz von den Zufallszahlen muss endlich sein, was f\"ur
    Verteilungen mit \emph{fat tails} nicht m�glich ist.
  
  \item Unabh�ngigkeit zwischen zwei Zufallszahlen wird durch eine
    Korrelation von null ausgedr\"uckt. Eine Interpretation in die
    Gegenrichtung ist jedoch nur f\"ur eine multivariate
    Normalverteilung richtig. Au�erdem h�lt das Argument auch nicht,
    wenn nur die Randverteilungen, nicht aber die gemeinsame Verteilung
    normalverteilt sind.
  
  \item Die Korrelation ist nicht invariant bez\"uglich monotoner
    Transformationen.
  \end{enumerate}
\end{itemize}

Um nun den Umgang mit diesen Problemen zu erleichtern, setzt man
vermehrt \emph{Copulae} ein. Mit Hilfe dieser kann die gemeinsame
Verteilung gesch�tzt werden. Au�erdem erlauben \emph{Copulae} die
Anwendung invariater Abh�ngigkeitsma�e, welche die Nachteile der
linearen Korrelation \"uberwinden.

Im n�chsten Kapitel folgt eine kurze Erkl�rung des Konzepts von
\emph{Copulae} und es werden die wesentlichsten Eigenschaften kurz
beschrieben. Danach soll kurz vorgestellt werden, welche Familien von
\emph{Copulae} es gibt und welche Eigenschaften diese besitzen.

Kapitel \ref{sec:fitmethoden} stellt den wesentlichen Teil dieser
Arbeit dar, n�mlich die unterschiedlichen Methoden, mit welchen eine
passende \emph{Copula} f\"ur Zeitreihen gefunden werden kann. Diese
Methoden werden durch Demonstration f\"ur den bivariaten Fall mit dem
Softwarepaket \emph{R} dargestellt.