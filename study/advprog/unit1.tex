\documentclass{article}
% \usepackage[utf8]{inputenc}
\usepackage[austrian]{babel}
\usepackage[round]{natbib}
\usepackage{graphicx}
\usepackage{hyperref}

\title{My First Document}
\author{Me (KH)}

\begin{document}

\maketitle

\begin{abstract}
  Das ist eine kurze Zusammenfassung.
\end{abstract}

\tableofcontents

\section{Whatever}

\section{Introduction}
\label{sec:intro}

This is      my    first \LaTeX{} file.

This is a new paragraph.

% Das ist sehr schön.
Das ist sehr sch\"on, ausserdem: \"a \`a \'a \v{a} \H{a}, au\ss{}erdem.
Wie sagte schon Pythagoras: $a^2 + b^2 = c^2$ oder eigentlich
\begin{math}\alpha^2 + \beta^2 = \gamma^2\end{math}.

\section{Results}

Und jetzt eine Gleichung:
$$ \int_{-\infty}^{\infty} \frac{1}{\sqrt{2\pi}} e^{-t^2/2} dt = 1 $$
oder auch
\begin{displaymath}
  \int_{-\infty}^{\infty} \left\{ \frac{1}{\sqrt{2\pi}} \right] e^{-t^2/2} dt = 1
\end{displaymath}

As already discussed in Section~\ref{sec:intro}, bla bla bla.

\subsection{Foo Bar}

{\Large UUU} {\tiny zzz} {\Huge AAA}

\textbf{Fett} und \textit{Italics} und \textsf{Sans Serif} und
\texttt{Typewriter}.

Zum betonen: \emph{text}.

\begin{itemize}
 \item Erstes.    Bla bla bla bla bla bla
  bla bla bla bla bla bla bla bla bla bla bla bla bla bla bla bla bla
  bla bla bla bla bla bla bla bla bla bla

  So mache ich einen Absatz.
 \item Zweites
\end{itemize}

\begin{enumerate}
 \item Erstes
  \begin{enumerate}
 \item Erstes
 \item Zweites
\end{enumerate}

 \item Zweites
\end{enumerate}

Und jetzt wollen wir die LVA-Teilnehmer kurz beschreiben.
\begin{description}
 \item[Reinhard] ist heute sp\"ater gekommen.  Bla bla bla bla bla bla
  bla bla bla bla bla bla bla bla bla bla bla bla bla bla bla bla bla
  bla bla bla bla bla bla bla bla bla bla

  So mache ich einen Absatz.
  
 \item[Alex] stellt immer subtile Fragen.
\end{description}

\begin{verbatim}
x <- 1
y <- 2
FOO <- function(x) { writeLines("Hello world.") }
\end{verbatim}

Und jetzt verweisen wir auf \cite{test:Walchhofer:2005}.

\section{Using R}

\begin{verbatim}
R> x <- runif(100)
R> summary(x)
    Min.  1st Qu.   Median     Mean  3rd Qu.     Max. 
0.003579 0.224800 0.529800 0.499800 0.731900 0.985700 
\end{verbatim}

Here we show how to include graphics.

\begin{figure}
  \centering
  \includegraphics[scale=0.5]{testbild.pdf}  
  \caption{Das ist ein Histogramm von 100 gleichverteilten Zufallszahlen.}
  \label{fig:testbild}
\end{figure}

For more info, see \url{http://statmath.wu-wien.ac.at/courses/}

\bibliographystyle{plainnat}
\bibliography{unit1}


\end{document}
