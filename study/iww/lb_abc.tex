\documentclass[a4paper]{article}
% PREAMBLE
\usepackage[austrian]{babel}
\usepackage[latin1]{inputenc}
\usepackage[authoryear,round]{natbib}
\usepackage{graphicx}
\usepackage{hyperref}

\title{Zusammenfassung: ABC der Leistungsbilanz}
\author{Stefan Theu�l, 0352689}

\begin{document}

\maketitle

\tableofcontents 

\begin{abstract}
Es soll kurz das Zusammenspiel Leistungsbilanz - Zahlungsbilanz -
Kapitalbilanz, welches in dem Artikel \textit{Das kleine ABC der
  Leistungsbilanz} erkl�rt wird, zusammengefasst werden.
\end{abstract}

\pagebreak

\section{Zahlungsbilanz, Leistungsbilanz}
\label{sec:el}

Aufgrund internationaler Konventionen werden alle Transaktionen
zwischen In- und Ausland �ber folgende Teilbilanzen zur Zahlungsbilanz
zusammengefasst:
\begin{description}
 \item[Leistungsbilanz] alle Transaktionen zwischen In- und Ausland,
   die Auswirkung auf Einkommen und Verbrauch haben.
 \item[Verm�gens�bertragung] 
 \item[Kapitalbilanz] deckt den gesamten Kapitalverkehr ab.
 \item[statistische Differenz] 
\end{description}

Als Vereinfachung wird die Zahlungsbilanz im Folgenden in 2
Komponenten zerlegt: Leistungsbilanz, Kapitalbilanz.

Leistungs- sowie Kapitalbilanz weisen die gleichen Salden auf nur mit
ungleichen Vorzeichen, da die Zahlungsbilanz immer ausgeglichen sein muss.  

Aus der bekannten Formel f�r das BIP kann man die Leistungsbilanz
ableiten:
$$ BIP = C + I + G + IM - EX $$
wobei $ IM - EX $ der Leistungsbilanz entspricht.

Davon abgeleitet kann man folgendes anschreiben:
$$ S - I = EX - IM + G - T $$
Leistungsbilanz und Staatsdefizit in Relation zum BIP summieren sich
auf die Differenz von Spar- und Investitionsquote.

Das wirft viele Fragen auf. Wie zum Beispiel: \textit{Verursacht ein
  Leistungsbilanzdefizit ein Staatsdefizit, oder umgekehrt?}.

Ein Leistungsbilanzdefizit kann aber auch Gr�nde haben, die von
Akteuren der Wirtschaftspolitik wenig bis gar nicht beeinflusst werden
k�nnen (z.B. LB-Defizit der USA - verursacht durch Dollarexport, da
der Dollar die wichtigste Handelsw�hrung der Welt ist).

�ber alle Volkswirtschaften ist die Leistungsbilanz ausgeglichen
(empirisch aber schwierig nachweisbar).

Auf die Kapitalbilanz wird in diesem Text nicht n�her eingegangen.

\section{Interpretationsversuche}

David Ricardo versuchte schon zu seinen Zeiten Unterschiede in der
Produktivit�t einzelner Volkswirtschaften aufzuzeigen. Er bediente
sich dabei seiner Theorie der komparativen Kostenvorteile.

Weiters gab es einen Ansatz, der Unterschiede in der Leistungsbilanz
auf die unterschiedliche Faktorausstattung einzelner L�nder
zur�ckf�hrte.

Weitere Erkl�rungsversuche:
Volkswirtschaften k�nnen in einer Periode mehr konsumieren als in
anderen und daher kurzfristig eine negative Leistungsbilanz
hervorrufen. Es m�ssen G�ter aus dem Ausland importiert werden, um die
inl�ndische Nachfrage zu decken.
Man sieht: das �konomische Prinzip des Angebot-Nachfrage
Gleichgewichts scheint auch hier fundamental zu sein.

Wichtig bei negativen Leistungsbilanzen ist, dass die importierten
G�ter nicht ausschlie�lich in den Konsum gehen, sondern an anderer
Stelle investiert werden sollen. Andernfalls kann es negative
Auswirkungen langfristiger Sicht haben.

\section{Empirische Erkenntnisse}

Der Text zeigt Beispiele der G7 L�nder (Exportquote). L�nder mit einem
gro�en Binnenmarkt (USA, Japan) weisen viel geringere Export- und
Importquoten auf als kleinere L�nder. 
Weiters wurde festgestellt, dass die Leistungsbilanz im Zeitablauf
schwankt.

Am Ende wurde auf das Leistungsbilanzdefizit der USA genauer
eingegangen. 
Es wurde viel dar�ber diskutiert, ob dadurch die Konkurrenzf�higkeit
der USA abnimmt. Manche Analysten orteten auch eine zu geringe
Sparquote (Schlagworte: Konsumausweitung, leichte Verf�gbarkeit von
Finanzierung, New Economy).

Die Ursachen k�nnen oder sollten eigentlich in der Kapitalbilanz
gesucht werden (ungebrochener Erwerb von amerikanischen
Verm�genswerten, aufgrund h�herer wirtschaftlicher Dynamik in den
USA).

Au�erdem muss ber�cksichtigt werden, dass der Dollar als
internationales Zahlungsmittel �u�erst beliebt ist. Das hei�t die USA
m�ssen f�r die notwendige Dollarliquidit�t sorgen. 
Insgesamt steigt der Druck auf die USA, eine polit�konomische gute
L�sung des Problems zu suchen. Bei einer darauf folgenden
Handelseinschr�nkung w�rde das negative Auswirkungen auf viele L�nder
haben, die zu den USA eine intensive Handelsverbindung halten.

\bibliographystyle{plainnat}

\end{document}