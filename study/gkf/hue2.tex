\documentclass[a4paper]{article}
% PREAMBLE
\usepackage[austrian]{babel}
\usepackage[latin1]{inputenc}
\usepackage[authoryear,round]{natbib}
\usepackage{graphicx}
\usepackage{hyperref}

\title{Vorteile von Banken - Liquidit�t}
\author{Stefan Theu�l, 0352689}

\begin{document}

\maketitle

\begin{abstract}
In dieser Zusammenhang von \cite{FreixasRochet} soll gezeigt werden,
dass in einer �konomie, in welcher Agenten individuellen
Liquidit�tsschocks ausgesetzt sind, die Marktallokation durch
Verwahrungsvertr�ge von Finanzintermedi�ren verbessert
werden kann.
\end{abstract}

\section{Idee der Liquidit�tsversicherung}
\label{sec:lv}

Die Idee dahinter ist, dass sich durch mehrere Haushalte ein
Liquidit�tspool bildet. Dieser Pool bietet eine gewisse
\textit{Versicherung} gegen Liquidit�tsschocks. So lange diese Schocks
nicht perfekt korreliert sind (Regelfall), steigt die Barreserve einer
Bank der Gr��e $N$ (interpretiert als eine Koalition von $N$
Haushalten) unterproportional zu $N$.
Jene Mittel, die �ber die Barreserve
hinausgehen, k�nnen f�r illiquide Investitionen verwendet werden, die
einen gewissen Profit einbringen k�nnen (\textit{fractional reserve system}). Ein Problem, das hier 
auftreten kann, ist, dass viele Einleger auf einmal ihre Einlage
abheben. Sei es
auch ohne Liquidit�tsgr�nden.

\section{Das Modell}

\textbf{Annahme:}~1~Gut - 3 Perioden �konomie, jeder Agent besitzt in $t = 0$
eine Einheit des Gutes, welches in den Perioden 1 oder 2 konsumiert
wird.
\newline
Der erwartete Nutzen soll nun maximiert werden. Betrachtet werden nun
folgende F�lle:

\begin{description}
 \item[Autarkie] Im einfachsten Fall gibt es keinen Handel zwischen
   den Agenten. Jeder Agent w�hlt ein Verh�ltnis I, welches in die
   illiquide Technologie investiert
   werden soll (perfekte Teilbarkeit ist angenommen). Bei Autarkie
   w�hlt jeder Konsument jenes Konsumprofil, welches seinen ex-ante
   Nutzen unter Nebenbedingungen maximiert.
  
 \item[Markt �konomie] Existiert ein Markt, so wird angenommen, dass
   Handel die Wohlfahrt erh�ht. Agenten k�nnen zum Zeitpunkt t=1 das
   Gut gegen eine risikolose Anleihe tauschen. Diese Marktallokation
   paretodominiert die Autarkie, da in diesem Fall keine Liquidation
   vorliegt. Jedoch kann gezeigt werden, dass diese Form von
   Allokation keine perfekte Versicherung gegen Liquidit�tsschocks
   darstellt.

 \item[Optimale Allokation durch Finanzintermedi�re] Aufgrund der
   Tatsache, dass niemand beobachten kann, wer fr�her konsumiert, kann
   die Marktallokation kein Pareto Optimum sein (der Anleihenmarkt
   selbst ist kein Garant f�r effiziente Risikostreuung). Das Pareto
   Optimum kann einfach �ber einen Finazintermedi�r implementiert
   werden. Dieser Intermedi�r stellt im Gegenzug f�r eine Einlage dem
   Individuum zum Zeitpunkt 1 bzw. 2 entsprechende Mittel zur
   Verf�gung. Der FI investiert in die illequide Technologie um seinen
   Verpflichtungen nachzukommen.
\end{description}


\bibliographystyle{plainnat}
\bibliography{references}


\end{document}