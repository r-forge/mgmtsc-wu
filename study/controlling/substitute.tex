Das Baugewerbe ist eines der Gewerbe, welches nicht oder nur kaum ersetzt werden kann. Weiters handelt es sich um eine sehr gro�e Branche in der nur Teilbereiche durch neue Technologien substituiert werden. Daher hat PORR nur sehr wenig Konkurrenz von Anbietern von Ersatzg�tern.

\subsubsection{Ersatzg�ter im Hochbau}
\label{sec:SubHochbau}
Fertigteilbau und Systembau sind erw�hnenswerte Substitutionsg�ter im Bereich der privaten Bauwirtschaft. Es hier muss aber erw�hnt werden, dass PORR als Unternehmen vor allem im Hochhaus-, Infrastruktur- und Tunnelbau angesiedelt ist.

Verst�rkt wird in Zukunft auf die Umweltvertr�glichkeit der benutzen Werkstoffe geschaut. Daher werden leichter, entsorgbare Stoffe bevorzugt. Auch die Energieeffizienz des Geb�udes wird durch die steigenden Energiekosten immer wichtiger. PORR hat diesen Trend erkannt und betreibt eine eigene F\&E Abteilung. Die Gefahr besteht aber, dass in Zukunft hochtechnologische Teilbereiche des Bau (z.B.: Isolierung) nur mehr durch kleinere spezialisierte Subunternehmen durchgef�hrt werden. 

Holz und Stahlbaukonstruktionen m�ssen auch noch erw�hnt werden.


\subsubsection{Substitutionsgefahr im Tiefbau}
\label{sec:SubTiefbau}

Der Infrastrukturbau umfasst unter anderem Strassen-, Eisenbahn-, Flughafen-, Tunnel- und Br�ckenbau. All diese Bereiche werden durch die steigende Mobilit�t immer st�rker und �fter benutzt. Die Ersatzg�ter im Tiefbau sind zum Gro�teil andere G�ter im Tiefbau. So k�nnte z.B. ein Tunnel durch eine Strasse oder eine Br�cke durch einen Tunnel ersetzt werden.

Eine Substitutionsgefahr gibt es nur im weiteren Sinne. F�hren oder Schiffe k�nnten Br�cken ersetzen, jedoch geht der Trend eher in die umgekehrte Richtung. Die vielleicht gr��te Gefahr geht von extrem ansteigenden �lpreisen aus. Falls die Mobilit�t nicht mehr leistbar w�re, braucht man auch keine Infrastruktur ausbauen. 



