\textbf{Hinweise zur Berechnung:}

\begin{itemize}
\item Umsatzerl�se = Absatzmenge $\times$ Verkaufspreis (inkl. USt)

\item Rabatt = Bruttoerl�se (exkl. USt) $\times ~  5~ \% $

\item Skonto = Zielerl�se $\times$ 1~\%

\item Vertreterprovision = Nettoerl�se $\times$ 3~\%

\item Lohn der Produktionsabteilung = Lohnkosten je Stunde $\times$ Stunden je St\"uck $\times$ Absatzmenge

\item var. Energiekosten = Energieeinheiten je St\"uck $\times$ Absatzmenge $\times$ Energiekosten je Einheit

\item Materialeinzelkosten = Materialkosten je St\"uck $/~1.2$ $\times$ Absatzmenge

\item Frachtkosten = Fracht je St\"uck $\times$ Absatzmenge

\item sonst. var. Herstellkosten = sonst. Fertigungsgemeinkosten $\times$ Absatzmenge der Produktgruppe $/$ gesamte Absatzmenge

\item kalkulatorische Abschreibung = $68,750 + 84,000 + 50,000~=~202,750$
  \begin{itemize}
  \item Maschine 1:\\
    Wiederbeschaffungswert $/$ kalk. Nutzungsdauer = $550,000 / 8~=~68,750$
  \item Maschine 2:\\
    Restwert Anfang 2007 $\times~30~\% $ = $400,000 \times 0.7 \times 30~\%~=~84,000$
  \item Maschine 4:\\
    Keine Abschreibung wegen Ver�u�erung.
  \item neue Anlage:\\
    Keine Abschreibung, weil die Inbetriebnahme erst 2008 erfolgt.
  \item Verpackungsmaschine:\\
    Wiederbeschaffungswert $/$ kalk. Nutzungsdauer = $500,000/10~=~50,000$
  \end{itemize}

\item kalkulatorische Zinsen des Anlageverm�gens~=~BW~Maschinen~Anfang~2007~$\times~8~\%$

\item kalkulatorische Zinses des Umlaufverm�gens~=~(AB~Material~-~EB~Material)~$/~2$ = $(150,000 + 50,000) / 2 $\\
  EB = AB + Zukauf - Verbrauch = $150,000 + 180,000 / 1.2 - 250,000$~(siehe Materialeinzelkosten)~$ =~50,000$
  
\item Versicherungen (enthalten in kalk. Wagnissen) =
  2 Monate von Pr�miensumme 48,000 und 10~Monate von Pr�miensumme 54,000\\
  = $48,000 / 12 \times 2 + 54,000 / 12 \times 10 = 53,000$

\item AfA laut Finanzbuchhaltung = $120,000 + 100,000 + 50,000 = 270,000$
  \begin{itemize}
  \item Maschine 1:\\
    Anschaffungswert $/$ Nutzungsdauer = $600,000 / 5~=~120,000$
  \item Maschine 2:\\
    Anschaffungswert $/$ Nutzungsdauer = $400,000 / 4~=~100,000$
  \item Maschine 4:\\
    Keine Abschreibung wegen Ver�u�erung (Anmerkung: Normalerweise m\"usste noch die Abschreibung bis zur Ver�u�erung ber\"ucksichtigt werden. Hierzu gibt es allerdings keine Angaben!)
  \item neue Anlage:\\
    Keine Abschreibung, weil die Inbetriebnahme erst 2008 erfolgt.
  \item Verpackungsmaschine:\\
    Anschaffungswert $/$ Nutzungsdauer = $500,000/10~=~50,000$
  \end{itemize}

\item Steuer = Unternehmensergebnis $\times~25~\%$
  
\end{itemize}
