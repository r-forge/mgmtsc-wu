\begin{table}[tbp]
  \centering
  \caption{Bewertung}
  \label{table:Bewertung}
  \begin{tabular}{|l|c|c|}
    \hline
    \begin{table}
  \centering
  \caption{Bewertung}
  \label{table:Bewertung}
  \begin{tabular}{|l|c|c|}
    \hline
    \begin{table}
  \centering
  \caption{Bewertung}
  \label{table:Bewertung}
  \begin{tabular}{|l|c|c|}
    \hline
    \begin{table}
  \centering
  \caption{Bewertung}
  \label{table:Bewertung}
  \begin{tabular}{|l|c|c|}
    \hline
    \input{bewertung.txt}
    \hline  
  \end{tabular}
\end{table}

    \hline  
  \end{tabular}
\end{table}

    \hline  
  \end{tabular}
\end{table}

    \hline  
  \end{tabular}
\end{table}

Tabelle~\ref{table:Bewertung} gibt eine \"Ubersicht \"uber die Bewertung der einzelnen Punkte der Branchenanalyse. Hierbei wurde nach dem Schulnotensystem bewertet.

Der Wettbewerb in der Branche ist zwar stark, jedoch kann sich die \emph{A. PORR AG} in diesem Konkurrenzkampf gut behaupten. In Zukunft wird die \emph{A. PORR AG} durch ihre Konzentration auf die osteurop�ischen M�rkte im Wettbewerbkampf profitieren.

Hinsichtlich dem Markteintritt neuer Wettbewerber besteht aufgrund des hohen Kapitalbedarfes und des n�tigen Know-Hows keine gr��ere Gefahr. In Zukunft k�nnte sich der derzeitige Marktvorteil (durch den fr\"uhen Markteintritt) in Osteuropa schw�cher auswirken, deshalb die schlechtere Prognose.

Bei der Substitutionsgefahr durch Ersatzprodukte ist in der Baubranche
eine differenzierte Betrachtung n�tig: Im Hochbau ist die Gefahr durch
Stahl-, Holz-, Ferteil-, Systembau usw. gegeben, im Tiefbau hingegen
gibt es kaum Alternativen.

Aufgrund der Gr��e der \emph{A. PORR AG} besitzt sie gegen\"uber ihren
Lieferanten eine sehr gro�e Verhandlungsmacht. Diese wird durch die
Ost-Expansion weiter ausgebaut.

Aufgrund des abzusehenden R\"uckgangs beim Kundenvolumen gibt es eine
Tendenz zu wachsender Verhandlungsmacht bei den Kunden. Dies betrifft
haupts�chlich den privaten Sektor, welcher gr��tenteils den Hochbau
betrifft. Weiters sind steigende Zinsen zu ber\"ucksichtigen, welche
sicherlich Auswirkungen auf Investitionen in der Bauindustrie haben
werden.

In Summe ist die Wettbewerbssituation f\"ur die \emph{A. PORR AG} zufriedenstellend. Unsere Prognose weist eine sehr leicht fallende Tendenz auf, allerdings muss hier gesagt werden, dass keine Gewichtung der einzelnen Teilbereiche vorgenommen wurde, beispielsweise fiel der Punkt Harmonisierung gleich schwer ins Gewicht wie das Marktwachstum in Osteuropa.